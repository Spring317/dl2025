\documentclass[hidelinks]{report}
\usepackage{graphicx} % Required for inserting images
\usepackage{amsmath}
\usepackage{siunitx}
\usepackage{placeins}
\usepackage{tikz}
\usetikzlibrary{automata, arrows, positioning}
\usepackage{float}
\usepackage{hyperref}
\usepackage{cite}
\usepackage[utf8]{inputenc}
\usepackage[T1]{fontenc}
\usepackage{xcolor,graphicx}
\setcounter{secnumdepth}{0}
\usepackage{titlesec}
\usepackage[left=1.5in,top=1in,bottom=1in,right=1in]{geometry}
\usepackage{rotating}
\usepackage{subcaption}
\usepackage{lipsum}
\usepackage{fancyhdr}
\usepackage{mathptmx}
\usepackage{listings}
\usepackage{booktabs}
\usepackage{caption}
\usepackage{subcaption}
\usepackage{adjustbox}
\usepackage{enumitem}
\usepackage{setspace}
\usepackage{multirow}
\usepackage{amsmath}
\usepackage{algorithm}
\usepackage{algpseudocode}
\usepackage{tikz}
\setcounter{tocdepth}{4}
\usepackage{amsfonts}
\usepackage{fvextra}


\DefineVerbatimEnvironment{MyVerbatim}{Verbatim}{breaklines=true}
\usepackage{pgfplots}
\pgfplotsset{compat=1.17}
\newcommand{\mynote}[2]{\fbox{\bfseries\sffamily\scriptsize{#1}} {\small\textsf{\emph{#2}}}}

\newcommand{\hieplnc}[1]{\textcolor{red}{\mynote{hieplnc}{#1}}}

\newcommand{\sontg}[1]{\textcolor{blue}{\mynote{sontg}{#1}}}

\definecolor{blue}{RGB}{31,56,100}

\usepackage{lipsum}% http://ctan.org/pkg/lipsum
\makeatletter
\def\@makechapterhead#1{%
  {
  \parindent \z@ \raggedright \normalfont   
    
    \ifnum \c@secnumdepth >\m@ne
        \huge\bfseries \thechapter.\ % <-- Chapter # (without "Chapter")
    \fi
    \interlinepenalty\@M
    #1\par\nobreak% <------------------ Chapter title
    \vskip 40\p@% <------------------ Space between chapter title and first paragraph
  }}
\makeatother


% Redefine the \thesection and \thesubsection representations
\renewcommand{\thesection}{\arabic{chapter}.\arabic{section}}
\renewcommand{\thesubsection}{\thesection.\arabic{subsection}}
\renewcommand{\thesubsubsection}{\thesubsection.\arabic{subsubsection}}

% Define a new counter for subsections
\newcounter{subsecindex}[section]
\renewcommand{\thesubsecindex}{\thesubsection%
  \ifnum\value{subsecindex}>0
    .\arabic{subsecindex}%
  \fi
}

% Redefine the \section command to include the index
\let\oldsection\section
\renewcommand{\section}[1]{%
  \setcounter{subsecindex}{0} % Reset subsection counter for each section
  \refstepcounter{section}%
  \oldsection{\thesection\hspace{0.5em}#1}%
}

% Redefine the \subsection command to include the index
\let\oldsubsection\subsection
\renewcommand{\subsection}[1]{%
  \refstepcounter{subsection}%
  \oldsubsection{\thesubsecindex\hspace{0.5em}#1}%
}

% Redefine the \subsubsection command to include the index
\let\oldsubsubsection\subsubsection
\renewcommand{\subsubsection}[1]{%
  \refstepcounter{subsubsection}%
  \oldsubsubsection{\thesubsecindex\hspace{0.5em}#1}%
}

\titleformat{\section}
  {\normalfont\LARGE\bfseries} % Adjust \Large to any size you prefer
  {\thesection}{3em}{}
\titleformat{\subsection}
  {\normalfont\Large\bfseries} % Adjust \Large to any size you prefer
  {\thesubsection}{3em}{}
\titleformat{\subsubsection}
  {\normalfont\Large\bfseries} % Adjust \Large to any size you prefer
  {\thesubsubsection}{3em}{}

\begin{document}
\pagenumbering{gobble}

\pdfbookmark[0]{Main Title}{maintitle}
\begin{titlepage}
    \begin{tikzpicture}[remember picture,overlay,inner sep=0,outer sep=0]
        \draw[black!70!black,line width=1.5pt]
            ([xshift=-0.65in,yshift=-1cm]current page.north east) coordinate (A) -- % Adjusted x-shift and y-shift
            ([xshift=0.65in,yshift=-1cm]current page.north west) coordinate (B) -- % Adjusted x-shift and y-shift
            ([xshift=0.65in,yshift=1cm]current page.south west) coordinate (C) -- % Adjusted x-shift and y-shift
            ([xshift=-0.65in,yshift=1cm]current page.south east) -- % Adjusted x-shift
            cycle;
    \end{tikzpicture}

    \begin{center}
    \begin{figure}
        \centering
        \huge \uppercase{university of science and technology of hanoi} \\ [1.5 cm]
    
        \filleft
        \includegraphics[width=0.7\linewidth]{images/usth.png}
    \end{figure}
    
    \textsc{\Large }\\[1cm]
    {\huge \bfseries \uppercase{LABWORK'S REPORT}}\\[1cm]

    {\large \bfseries 2440053 - Dao Xuan Quy  } \\ [0.5cm]
    {\huge \bfseries \uppercase{Deep Learning}}\\[1cm]
    
    % Title
    \rule{\linewidth}{0.3mm} \\[0.4cm]
    { \Huge \bfseries\color{blue}  Labwork 2: Implement Linear Regression}
    \rule{\linewidth}{0.3mm} \\[0.7cm]
    
    \large Academic Year: 2024-2026
    \end{center}

\end{titlepage}

\newpage
\pagenumbering{roman}
\noindent \Large \tableofcontents

\newpage
\listoffigures

\newpage
\listoftables

\thispagestyle{empty}
\newpage

\chapter{Introduction}
\pagenumbering{arabic}

\hspace{5mm} In this labwork I implement linear regression based on the gradient descend on two variable in the previous labwork and dicuss the result.

 
\chapter{Implementation}
\section{Source code}
\noindent This is my implementation of gradient descend for 2 variables:

\begin{verbatim}
   def gradient_descend_2d(old_w0, old_w1, x, y, lr, iterations = 1000): 
    loss = []
    # xs = []
    old_f = None
    count = 0
    w0 = old_w0
    w1 = old_w1
    while True:
        f_prime = w1 * x + w0 - y
        w0 = w0 - lr*f_prime
        w1 = w1 - lr*f_prime  
        f = 1/2 * (w1* x + w0 - y)**2
        loss.append(f)
        # xs.append([w0, w1])
        print(f"time: {count} w0: {w0}, w1: {w1}, f(x): {f}")
        
        if (old_f is not None and abs(old_f - f) < 1):
            break
        
        old_f = f
        count += 1    
    return loss, w0, w1 
\end{verbatim}

And the linear regression algorithm should looks like this:

\begin{verbatim}
    def linear_regression(data):
    w0 = 0.0
    w1 = 1.0
    lr = 0.0001
    
    av_loss = []
    for x,y in zip(data[0], data[1]):
        loss, w0, w1 = gradient_descend_2d(w0, w1, x, y, lr)
        print(loss)
        avg_loss = sum(loss)/len(loss)
        av_loss.append(avg_loss)
    print(av_loss)
    return av_loss, w0, w1

\end{verbatim}

\section{Dataset}
The dataset used in this implementation is shown as follows:
\begin{table}[h!]
\centering
\begin{tabular}{cc}
\toprule
\textbf{X} & \textbf{Y} \\
\midrule
10 & 55 \\
20 & 80 \\
40 & 100 \\
60 & 120 \\
80 & 150 \\
\bottomrule
\end{tabular}
\caption{Dataset}
\end{table}

\chapter{Results}
\section{Setup}
The linear regression is setup as follow:
\begin{itemize}
    \item Iniital: w0 = 0.0, w1 = 1.0
    \item learning rate = $10^{-3}$
    \item threshold = 1
\end{itemize}

\section{Results:}
After 107 iteration, the two weight values are w0 = 1.0002129597767853 and w1 = 2.000212959776786 and the regression line is as follow:

\begin{figure}[h!]
    \centering
    \includegraphics[width=0.7\linewidth]{fig/linear_line.png}
    \caption{Linear Regression Line}
    \label{fig:enter-label}
\end{figure}

\begin{itemize}
    \item The learning rate affects convergence speed and stability.
    \item With a small learning rate (e.g., 0.0001), the model converges slowly but stably.
    \item A large learning rate (e.g., 0.1) may lead to faster convergence but risks overshooting.
    \item A suitable learning rate balances speed and accuracy
\end{itemize}

\chapter{Conclusion}
In this lab, I implemented linear regression using the gradient descent algorithm on a dataset with two variables. The goal was to optimize the weight parameters \( w_0 \) and \( w_1 \) to minimize the mean squared error between the predicted and actual values. Through iterative updates, the model successfully converged to a linear regression line that fits the data.



\end{document}